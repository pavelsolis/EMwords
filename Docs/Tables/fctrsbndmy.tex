\documentclass[a4paper,12pt]{article}
\usepackage[labelsep=period,labelfont=bf]{caption}
\usepackage{multirow}
\usepackage{booktabs}
\usepackage{threeparttable}
\usepackage{pdflscape}
\usepackage{tabularx}
\usepackage{afterpage}
\usepackage[margin=1in]{geometry}
\input{../Settings/macros_global}			   % Personalized commands
\input{../Settings/macros_local}			    % Personalized commands
%\pagestyle{empty}

\begin{document}
	\afterpage{
	\begin{normalsize}
		\begin{landscape}
			\begin{table}
				\begin{center}
					\caption{Response of Monthly Bonos Flows to Target and Path Surprises} \label{tab:fctrsbndmy}
					\begin{threeparttable}
						\estauto{../Tables/f_fctrsbndmy.tex}{6}		% 8
						\tabnote{This table shows the coefficient estimates in regressions of different categories of Mexican bond inflows on target and path surprises. Inflows are obtained as the change in the holdings of Mexican bonds. All flows are expressed in billions of Mexican pesos. The surprises are equal to the estimated value (as explained in the main text) if there was a monetary policy announcement in the respective month and zero otherwise. The lag order for each flow category is selected using the Bayesian information criterion. The sample period goes from January 2011 to \lastobsflwbdm. All regressions include a constant. Robust standard errors are shown in parentheses. *, **, *** asterisks respectively indicate significance at the 10\%, 5\% and 1\% level.}
					\end{threeparttable}
				\end{center}
			\end{table}
		\end{landscape}
	\end{normalsize}
	}
\end{document}