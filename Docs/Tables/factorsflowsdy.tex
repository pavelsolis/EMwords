\documentclass{article}
\usepackage{booktabs}
\usepackage{tabularx}
\usepackage[margin=1in]{geometry}
\usepackage{threeparttable}
\usepackage{pdflscape}
\begin{document}

\begin{landscape}
\begin{table}[tbp] \centering
\newcolumntype{C}{>{\centering\arraybackslash}X}
\begin{threeparttable}
\caption{The Response of Portfolio Flows to Target and Path Factors}
\label{tab:factorsflows}
{\normalsize
\begin{tabularx}{\linewidth}{lCCCCCCCC}

\toprule
&\multicolumn{4}{c}{Inflows}&\multicolumn{4}{c}{Outflows} \tabularnewline \cmidrule(lr){2-5}\cmidrule(lr){6-9} &\multicolumn{1}{c}{T-Bonds}&\multicolumn{1}{c}{Agency}&\multicolumn{1}{c}{Non-U.S.}&\multicolumn{1}{c}{Non-U.S.}&\multicolumn{1}{c}{T-Bonds}&\multicolumn{1}{c}{Agency}&\multicolumn{1}{c}{Non-U.S.}&\multicolumn{1}{c}{Non-U.S.} \tabularnewline &\multicolumn{1}{c}{T-Notes}&\multicolumn{1}{c}{Bonds}&\multicolumn{1}{c}{Bonds}&\multicolumn{1}{c}{Stocks}&\multicolumn{1}{c}{T-Notes}&\multicolumn{1}{c}{Bonds}&\multicolumn{1}{c}{Bonds}&\multicolumn{1}{c}{Stocks} \tabularnewline \cmidrule(lr){2-2}\cmidrule(lr){3-3}\cmidrule(lr){4-4}\cmidrule(lr){5-5}\cmidrule(lr){6-6}\cmidrule(lr){7-7}\cmidrule(lr){8-8}\cmidrule(lr){9-9} \tabularnewline
%\midrule \addlinespace[\belowrulesep]
Target Factor&-0.055&0.013&-0.058*&-0.018***&-0.013&0.012&-0.015&-0.015*** \tabularnewline
&(0.059)&(0.018)&(0.029)&(0.005)&(0.024)&(0.008)&(0.032)&(0.005) \tabularnewline
Path Factor&0.102&0.057**&0.075&-0.006&0.132*&0.003&0.033&-0.003 \tabularnewline
&(0.106)&(0.025)&(0.095)&(0.011)&(0.068)&(0.014)&(0.035)&(0.011) \tabularnewline
\midrule Lags&0&1&1&3&0&3&3&3 \tabularnewline
Observations&107&107&107&107&107&107&107&107 \tabularnewline
R-squared&0.027&0.236&0.136&0.603&0.028&0.227&0.256&0.594 \tabularnewline
\bottomrule \addlinespace[0cm]

\end{tabularx}
\begin{tablenotes}[para,flushleft]
\footnotesize \textit{Notes:} This table shows the coefficient estimates in regressions of different categories of portfolio inflows and outflows on the target and path factors obtained from daily data, as explained in the main text. Inflows are sales of securities by Mexican to U.S. investors. Outflows are purchases of securities by Mexican from U.S. investors. All flows are expressed in billions of U.S. dollars. The factors are equal to the estimated value if there was a monetary policy announcement in the respective month and zero otherwise. The lag order for each flow category is selected using the Bayesian information criterion. The sample period is January 2011 to November 2019. All regressions include a constant. Robust standard errors are shown in parentheses. *, **, *** asterisks respectively indicate significance at the 10\%, 5\% and 1\% level.
\end{tablenotes}
}
\end{threeparttable}
\end{table}
\end{landscape}
\end{document}
