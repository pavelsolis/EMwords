\documentclass[a4paper,12pt]{article}
\usepackage[labelsep=period,labelfont=bf]{caption}
\usepackage{multirow}
\usepackage{booktabs}
\usepackage{threeparttable}
\usepackage{pdflscape}
\usepackage{tabularx}
\usepackage[margin=1in]{geometry}
\input{../Settings/macros_global}			   % Personalized commands
\input{../Settings/macros_local}			    % Personalized commands
%\pagestyle{empty}

\begin{document}
	\begin{normalsize}
		\begin{landscape}
			\begin{table}
				\begin{center}
					\caption{Response of Asset Prices to Daily Target and Path Surprises} \label{tab:fctrsdyfxyc}
					\begin{threeparttable}
						\estauto{../Tables/f_fctrsdyfxyc.tex}{10}
						\tabnote{The first column for each dependent variable shows the coefficient estimates in regressions of \textit{intraday} yield changes or exchange rate (FX) returns on target surprises; the second column adds path surprises as a regressor. Target and path surprises are obtained from \textit{daily} data. Intraday changes are calculated starting 10 minutes before to 20 minutes after a monetary policy announcement. The sample includes all regular monetary policy announcements starting on January 2011 for the exchange rate, on January 2013 for 2- 10- and 30-year yields, and on December 2014 for 5-year yields; the sample ends on \lastobs{} in all cases. Figures are expressed in basis points. Heteroskedasticity-robust standard errors are shown in parentheses. *, **, *** asterisks respectively indicate significance at the 10\%, 5\% and 1\% level.}
					\end{threeparttable}
				\end{center}
			\end{table}
		\end{landscape}
	\end{normalsize}
\end{document}