\begin{abstract}
	This paper analyzes the price and quantity effects of monetary policy statements in an emerging market economy. Surprises in monetary policy are identified using intraday data on asset prices around monetary policy announcements in Mexico. I find that asset prices and the portfolio flows of domestic and foreign investors respond not only to changes in the policy rate but to information about its future path communicated via statements. The ability to manage expectations about future policy via statements is thus not exclusive to central banks in advanced economies and does not require the zero lower bound to be binding. 
	
	% Find the word count in the terminal: pbpaste | wc -w
	\vspace{.5cm}
	\noindent \textit{Keywords}: Monetary policy surprises, exchange rate, yield curve, portfolio flows, high-frequency data, event study.
	
	\noindent \textit{JEL Classification}: E52, E58, E43, F31, G14. 
	\vfill
	
	\pagebreak
\end{abstract}

